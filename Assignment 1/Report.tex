\documentclass[12pt]{article}
\usepackage[margin=0.8in,bottom=0.4in,top=0.5in]{geometry}
\usepackage{amsmath}
\usepackage{indentfirst}
\usepackage{listings}
\usepackage{color}
\usepackage{scrextend}
\usepackage{expl3,xparse}

\definecolor{gray}{rgb}{0.8,0.8,0.8}
\definecolor{mauve}{rgb}{0.58,0,0.82}
\definecolor{cream}{rgb}{1.0, 0.99, 0.6}

\lstset{language=C,
    basicstyle=\ttfamily,
    keywordstyle=\color{blue}\ttfamily,
    stringstyle=\color{mauve}\ttfamily,
    commentstyle=\color{green}\ttfamily,
    morecomment=[l][\color{magenta}]{\#}
    showstringspaces=false,
    morekeywords={include, printf},
    backgroundcolor=\color{white},
    showstringspaces=false,
    numbers=left,
    stepnumber=1,
    breaklines=true,
}

\title{\underline{\textbf{Assignment 1 Report}}}
\author{\textbf{Anoop (2015CS10265)}}

\begin{document}
\pagenumbering{gobble}
\maketitle

\section*{\underline{Part 1 - System Call Tracing}}
\subsection*{Printing the Trace}
\begin{addmargin}[0.7in]{0in}
\begin{lstlisting}[language=C,firstnumber=235,caption=Modification in ``{\ttfamily{syscall}}" function - {\ttfamily{syscall.c}},frame=single]
if(num > 0 && num < NELEM(syscalls) && syscalls[num]) {
\end{lstlisting}

\vspace{-\baselineskip}
\vspace{0.2mm}
\begin{lstlisting}[language=C,firstnumber=236,frame=single,backgroundcolor=\color{cream}]
    track_syscalls(curproc);
\end{lstlisting}

\vspace{-\baselineskip}
\vspace{0.2mm}
\begin{lstlisting}[language=C,firstnumber=237,frame=single]
    curproc->tf->eax = syscalls[num]();
} else {
    cprintf("%d %s: unknown sys call %d\n",
    curproc->pid, curproc->name, num);
    curproc->tf->eax = -1;
}
\end{lstlisting}
\end{addmargin}

\vspace{5mm}
\begin{addmargin}[0.1in]{0in}
    At line $236$ in Listing 1, the function {\ttfamily{track\_syscalls}} is used to print the syscalls that are issued only if the {\ttfamily{toggle\_state}} is set to $1$.
\end{addmargin}
\vspace{5mm}

\begin{addmargin}[0.7in]{0in}
\begin{lstlisting}[language=C,firstnumber=139,caption={\ttfamily{track\_syscalls}} function implementation - {\ttfamily{syscall.c}},frame=single,backgroundcolor=\color{cream}]
int syscall_count[24] = {0};
\end{lstlisting}

\vspace{-\baselineskip}
\vspace{0.2mm}
\begin{lstlisting}[language=C,firstnumber=140,frame=single,backgroundcolor=\color{cream}]
// Syscall identification
void
track_syscalls(struct proc *p)
{   
    ++syscall_count[p->tf->eax - 1];
    if(toggle_state == 1)
    {
        switch(p->tf->eax)
        {
            case 1:
                cprintf("%s %d\n","sys_fork",syscall_count[p->tf->eax - 1]);
                break;
            case 2:
                .
                .
\end{lstlisting}

\begin{lstlisting}[language=C,firstnumber=221,frame=single,backgroundcolor=\color{cream}]
            default:
                cprintf("Unknown syscall\n");
                break;
        }
    }
}
\end{lstlisting}
\end{addmargin}

\vspace{5mm}
\begin{addmargin}[0.1in]{0in}
    {\ttfamily{track\_syscalls}} function implementation uses a global array ({\ttfamily{syscall\_count}}) to maintain the count of the syscalls issued. The function uses {\ttfamily{switch-case}} statements to print appropriate syscall count when called only if the {\ttfamily{toggle\_state}} is set to $1$.
\end{addmargin}

\subsection*{Toggling the Printing of Trace}
\begin{addmargin}[0.7in]{0in}
\begin{lstlisting}[language=C,firstnumber=93,caption={\ttfamily{sys\_toggle}} function implementation - {\ttfamily{sysproc.c}},frame=single,backgroundcolor=\color{cream}]
// toggles the system trace (1 - ON | 0 - OFF)
int toggle_state = 1;
int
sys_toggle(void)
{
    toggle_state = toggle_state == 0 ? 1 : 0;
    return toggle_state;
}
\end{lstlisting}
\end{addmargin}

\vspace{5mm}
\begin{addmargin}[0.1in]{0in}
    For the {\ttfamily{sys\_toggle}} system call implementation, a global variable ({\ttfamily{toggle\_state}}) has been used to indicate the state of the system trace. Corresponding additions of definitions have been done in files - {\ttfamily{user.h, syscall.h, usys.S}}. An {\ttfamily{if}} statement has been added to {\ttfamily{syscall.c}} to check the {\ttfamily{toggle\_state}} before printing the trace (see code - Listing 1). {\ttfamily{user\_toggle.c}} file has been added to test the toggle system call implementation.
\end{addmargin}

\section*{\underline{Part 2 - {\ttfamily{sys\_add}} System Call}}
\begin{addmargin}[0.7in]{0in}
\begin{lstlisting}[language=C,firstnumber=102,caption={\ttfamily{sys\_add}} function implementation - {\ttfamily{sysproc.c}},frame=single,backgroundcolor=\color{cream}]
// add two numbers
int
sys_add(void)
{
    int a;
    int b;

    if(argint(0, &a) < 0)
    return -1;
    if(argint(1, &b) < 0)
    return -1;
    return a + b;
}
\end{lstlisting}
\end{addmargin}

\begin{addmargin}[0.1in]{0in}
    For the {\ttfamily{sys\_add}} system call implementation, a function that takes two arguments and returns the sum is used. The function checks for two arguments and if not present returns $-1$ as an error code. Corresponding additions of definitions have been done in files - {\ttfamily{user.h, syscall.h, usys.S, syscall.c}}.
\end{addmargin}

\section*{\underline{Part 3 - {\ttfamily{sys\_ps}} System Call}}
\begin{addmargin}[0.7in]{0in}
\begin{lstlisting}[language=C,firstnumber=270,caption={\ttfamily{sys\_ps}} function implementation - {\ttfamily{proc.c}},frame=single,backgroundcolor=\color{cream}]
// list all processes
int
sys_ps(void)
{
    struct proc *p;
    
    acquire(&ptable.lock);

    for(p = ptable.proc; p < &ptable.proc[NPROC]; p++){
        if(p->pid != 0 && p->state != ZOMBIE){
            cprintf("pid:%d name:%s\n",p->pid,p->name);
        }
    }

    release(&ptable.lock);
    return 0;
}
\end{lstlisting}
\end{addmargin}

\vspace{5mm}
\begin{addmargin}[0.1in]{0in}
    For the {\ttfamily{sys\_ps}} system call implementation, a function that loops over the {\ttfamily{ptable}} has been used. The function locks the {\ttfamily{ptable}} during the loop and prints the pid and name of process with non-zero pid. The zombie processes are ignored in this implementation. Corresponding additions of definitions have been done in files - {\ttfamily{user.h, syscall.h, usys.S, syscall.c}}. Also an {\ttfamily{extern}} definition has been added to {\ttfamily{sysproc.c}}.
\end{addmargin}

\end{document}