\documentclass[12pt]{article}
\usepackage[margin=0.8in,bottom=0.4in,top=0.5in]{geometry}
\usepackage{amsmath}
\usepackage{indentfirst}
\usepackage{listings}
\usepackage{color}
\usepackage{scrextend}
\usepackage{expl3,xparse}

\definecolor{gray}{rgb}{0.8,0.8,0.8}
\definecolor{mauve}{rgb}{0.58,0,0.82}
\definecolor{cream}{rgb}{1.0, 0.99, 0.6}

\lstset{language=C,
    basicstyle=\ttfamily,
    keywordstyle=\color{blue}\ttfamily,
    stringstyle=\color{mauve}\ttfamily,
    commentstyle=\color{green}\ttfamily,
    morecomment=[l][\color{magenta}]{\#}
    showstringspaces=false,
    morekeywords={include, printf},
    backgroundcolor=\color{white},
    showstringspaces=false,
    numbers=left,
    stepnumber=1,
    breaklines=true,
}

\title{\underline{\textbf{Assignment 2 Report}}}
\author{\textbf{Anoop (2015CS10265)}}

\begin{document}
\pagenumbering{gobble}
\maketitle

\section*{\underline{Part 1 - Add and print priority of the process}}
\subsection*{Modify {\ttfamily{sys\_ps}} to print priority}
\begin{addmargin}[0.3in]{0in}
    To implement this, process structure has been changed in \textbf{proc.h} file [Line 44]. Default priority has been set to 5 in \textbf{proc.c} file [Line 90]. To print the priority while listing processes, changes are made to \textbf{ps} function in \textbf{proc.c} file.
\end{addmargin}

\subsection*{Add {\ttfamily{sys\_setpriority}} system call}
\begin{addmargin}[0.3in]{0in}
\begin{lstlisting}[language=C,firstnumber=125,caption={\ttfamily{sys\_setpriority}} function implementation - {\ttfamily{sysproc.c}},frame=single,backgroundcolor=\color{cream}]
extern int setpriority(int,int);

// set priority
int
sys_setpriority(void)
{
    int pid;
    int priority;

    if(argint(0, &pid) < 0)
    return -1;
    if(argint(1, &priority) < 1 && argint(1, &priority) > 20){
    cprintf("Error\n");
    return -1;
    }
    return setpriority(pid,priority);
}
\end{lstlisting}
\begin{lstlisting}[language=C,firstnumber=303,caption={\ttfamily{setpriority}} function implementation - {\ttfamily{proc.c}},frame=single,backgroundcolor=\color{cream}]
// set priority
int
setpriority(int pid,int priority)
{
  struct proc *p;
  
  acquire(&ptable.lock);

  for(p = ptable.proc; p < &ptable.proc[NPROC]; p++){
    if(p->pid == pid){
      p->priority = priority;
      break;
    }
  }

  release(&ptable.lock);
  return pid;
}
\end{lstlisting}
\end{addmargin}

\vspace{5mm}
\begin{addmargin}[0.3in]{0in}
    For the {\ttfamily{sys\_setpriority}} system call implementation, a function takes that takes two arguments - PID and Priority and returns pid if successful or -1 on failure is used. {\ttfamily{ptable}} is locked while setting the priority. It report ``Error" on passing invalid priority (\textgreater{} 1 or \textless{} 20).
\end{addmargin}

\section*{\underline{Part 2 - Implement Priority Scheduler}}
\begin{addmargin}[0.3in]{0in}
To implement Priority based Round Robin, \textbf{scheduler} function has been updated in \textbf{proc.c} file. Lines 415 - 423 find the process with highest priority in the ptable. Lines 435 - 447  ensure the Round Robin fashion running of processes.
\end{addmargin}

\section*{\underline{Part 3 - Starvation}}
\subsection*{Add {\ttfamily{sys\_getpriority}} system call}
\begin{addmargin}[0.3in]{0in}
\begin{lstlisting}[language=C,firstnumber=143,caption={\ttfamily{sys\_getpriority}} function implementation - {\ttfamily{sysproc.c}},frame=single,backgroundcolor=\color{cream}]
extern int getpriority(int);

// get priority
int
sys_getpriority(void)
{
    int pid;

    if(argint(0, &pid) < 0)
        return -1;
    return getpriority(pid);
}
\end{lstlisting}
\begin{lstlisting}[language=C,firstnumber=322,caption={\ttfamily{getpriority}} function implementation - {\ttfamily{proc.c}},frame=single,backgroundcolor=\color{cream}]
// get priority
int
getpriority(int pid)
{
    struct proc *p;
    int priority = -1;
    
    acquire(&ptable.lock);

    for(p = ptable.proc; p < &ptable.proc[NPROC]; p++){
        if(p->pid == pid){
            priority = p->priority;
            break;
        }
    }
    release(&ptable.lock);
    if(priority == -1)
        cprintf("PID not found");
    return priority;
}
\end{lstlisting}
\end{addmargin}

\vspace{5mm}
\begin{addmargin}[0.3in]{0in}
For the {\ttfamily{sys\_getpriority}} system call implementation, a function takes that takes two arguments - PID and returns priority if successful or -1 on failure is used. {\ttfamily{ptable}} is locked while setting the priority. It report ``PID not found" on passing PID which does not exist.
\end{addmargin}
\subsection*{Handle starvation}
\begin{addmargin}[0.3in]{0in}
To handle starvation a counter is implemented which counts the number of context switches. It adds 1 to priority after counter reaches 50 and resets counter to 0. This has been implemented in \textbf{scheduler} function in \textbf{proc\_3.c} file [Line 449 - 460].
\end{addmargin}

\end{document}